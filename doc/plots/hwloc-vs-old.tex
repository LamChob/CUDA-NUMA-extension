\documentclass{scrartcl}

\usepackage[paper=a4paper,left=30mm,right=20mm,top=25mm,bottom=30mm]{geometry}
\usepackage[utf8]{inputenc}
\usepackage[T1]{fontenc}
\usepackage{lmodern}
\usepackage[ngerman, english]{babel}
\usepackage{amsmath}
\usepackage{booktabs}		% toprule,midrule,...
\usepackage{pdfpages}
\usepackage{pgfplotstable}
\usepackage{pgfplots}
\usepackage{graphicx}
\usepackage{listings}            % for code listings
\usepackage{courier}
\usepackage{color}

\definecolor{mygreen}{rgb}{0,0.6,0}
\definecolor{mygray}{rgb}{0.5,0.5,0.5}
\definecolor{mymauve}{rgb}{0.58,0,0.82}

%\pgfplotsset{compat=1.13}
\selectlanguage{english}

\lstset{ %
  backgroundcolor=\color{white},   % choose the background color; you must add \usepackage{color} or \usepackage{xcolor}
  basicstyle=\scriptsize\ttfamily,        % the size of the fonts that are used for the code
  captionpos=b,                    % sets the caption-position to bottom
  frame=single,                    % adds a frame around the code
  keepspaces=true,                 % keeps spaces in text, useful for keeping indentation of code (possibly needs columns=flexible)
  keywordstyle=\color{blue},       % keyword style
  language=C,                      % the language of the code
  numbers=left,                    % where to put the line-numbers; possible values are (none, left, right)
  numbersep=5pt,                   % how far the line-numbers are from the code
  numberstyle=\tiny\color{mygray}, % the style that is used for the line-numbers
  rulecolor=\color{black},         % if not set, the frame-color may be changed on line-breaks within not-black text (e.g. comments (green here))
  stepnumber=2,                    % the step between two line-numbers. If it's 1, each line will be numbered
  stringstyle=\color{mymauve},     % string literal style
  showstringspaces=false,
  tabsize=2,                       % sets default tabsize to 2 spaces
  title=\lstname,                  % show the filename of files included with \lstinputlisting; also try caption instead of title
  aboveskip=10pt,
  belowskip=10pt
}

\title{Exercise 4}
\author{Dennis Rieber, Dominik Zosgornik}
%\date{24. October 2015}
\date{\today}
\begin{document}

%\maketitle
%\tableofcontents

\begin{figure}[hbtp]
\begin{tikzpicture}
\begin{axis}[
    width=\textwidth,
    height=12cm,
    xlabel={Sample Size (MB)},
    ylabel={Bandwidth (GB/s)},
    grid=major,
    legend entries={hwloc 1T, first touch1T, hwloc 8T, first touch 8T},
   % ymode=log,
    xmode = log,
    legend pos=north west,
    legend cell align=left
]
\addplot[thick,mark=x,blue,each nth point={8}] table[x=size, y=bw] {NUMA-Migration/data/migrate_hwloc.dat};
\addplot[thick,mark=o,red,each nth point={8}] table[x=size, y=bw] {NUMA-Migration/data/migrate_old.dat};
\addplot[thick,mark=x,black,each nth point={8}] table[x index={0}, y index={2}, skip first n=8] {NUMA-Migration/data/migrate_hwloc.dat};
\addplot[thick,mark=o,green,each nth point={8}] table[x index={0}, y index={2}, skip first n=8] {NUMA-Migration/data/migrate_old.dat};

\end{axis}
\end{tikzpicture}
\caption{Migration performance, depending on problem size}
\label{fig:ex04}
\end{figure}
\begin{figure}[hbtp]
	\begin{tikzpicture}
	\begin{axis}[
	width=\textwidth,
	height=12cm,
	xlabel={num. of threads},
	ylabel={Bandwidth (GB/s)},
	grid=major,
	legend entries={hwloc, first touch},
	% ymode=log,
	legend pos=south east,
	legend cell align=left,
	]
	\addplot[mark=x,black,thick] table[x index={1}, y index={2}, skip first n=65] {NUMA-Migration/data/migrate_hwloc.dat};
	\addplot[mark=o,green,thick] table[x index={1}, y index={2}, skip first n=65] {NUMA-Migration/data/migrate_old.dat};
	\end{axis}
	\end{tikzpicture}
	\caption{Migration performance, depending on number thread. Sample Size is 8GB. Num. of threads also represents number of distribution domains. (4 threads mean, data is also distrubuted to 4 domains)}
\end{figure}

\begin{figure}[hbtp]
	\begin{tikzpicture}
	\begin{axis}[
	width=\textwidth,
	height=12cm,
	xlabel={Sample Size (MB)},
	ylabel={Bandwidth (GB/s)},
	grid=major,
	legend entries={hwloc 1T, first touch1T, hwloc 8T, first touch 8T},
	% ymode=log,
	xmode = log,
	legend pos=north west,
	legend cell align=left
	]
	\addplot[thick,mark=x,blue,each nth point={8}] table[x index={0}, y index={2}] {NUMA-Migration/data/migrate_hwloc_o3.dat};
	\addplot[thick,mark=o,red,each nth point={8}] table[x index={0}, y index={2}] {NUMA-Migration/data/migrate_old_o3.dat};
	\addplot[thick,mark=x,black,each nth point={8}] table[x index={0}, y index={2}, skip first n=8] {NUMA-Migration/data/migrate_hwloc_o3.dat};
	\addplot[thick,mark=o,green,each nth point={8}] table[x index={0}, y index={2}, skip first n=8] {NUMA-Migration/data/migrate_old_o3.dat};
	
	\end{axis}
	\end{tikzpicture}
	\caption{Optimized migration performance, depending on problem size}
	\label{fig:ex04}
\end{figure}
\begin{figure}[hbtp]
	\begin{tikzpicture}
	\begin{axis}[
	width=\textwidth,
	height=12cm,
	xlabel={num. of threads},
	ylabel={Bandwidth (GB/s)},
	grid=major,
	legend entries={hwloc, first touch},
	% ymode=log,
	legend pos=south east,
	legend cell align=left,
	]
	\addplot[mark=x,black,thick] table[x index={1}, y index={2}, skip first n=65] {NUMA-Migration/data/migrate_hwloc_o3.dat};
	\addplot[mark=o,green,thick] table[x index={1}, y index={2}, skip first n=65] {NUMA-Migration/data/migrate_old_o3.dat};
	\end{axis}
	\end{tikzpicture}
	\caption{Optimized migration performance, depending on number thread. Sample Size is 8GB. Num. of threads also represents number of distribution domains. (4 threads mean, data is also distrubuted to 4 domains)}
\end{figure}
\end{document}
